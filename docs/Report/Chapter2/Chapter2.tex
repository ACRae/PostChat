
\chapter{State of the art} 
\label{ch:Chapter2}
\vfill \newpage

In today's digital landscape, several messaging apps compete to provide users with a platform for connecting and sharing personalized messages. This chapter aims to compare PostChat with messaging and postcard apps such as \textit{\cite{WhatsApp}}, \textit{\cite{Telegram}}, and \textit{\cite{MyPostcard}}, highlighting PostChat's state-of-the-art qualities and distinct features over these apps. By analyzing their design, usability, and user experience, we can gain insights into PostChat's unique position, distinct idea, and what it borrows from its contemporaries.


\textit{PostChat}, much like WhatsApp or Telegram, prioritizes a user-friendly design and intuitive navigation. Its interface is simple and streamlined, ensuring a seamless postcard creation and sending experience for users of all ages. Also like those apps, it allows to create groups but instead of sending messages it sends postcards.

Unlike WhatsApp or Telegram, PostChat is focused on sending postcards and has a set of simple and straight forward drawing tools, high quality postcard templates and AI assisting tools that distinguish it from those other apps. 

MyPostcard has a different approach as it allows users to fully customize a postcard via a digital interface but send the postcard via physical mail, that is, no chat groups, no way to find friends in the service, just plain address form filling. PostChat approach allows for both formats as it send the digital postcards and can export them in order to printed onto physical paper.

In conclusion \textit{PostChat} borrows elements from different apps and makes a unique 
format for itself opening doors for a new way of communication and a new possible business 
opportunity.