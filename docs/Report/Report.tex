%/////////////////////////////////////////////////////////////////////
%
% LaTeX definitions:     Ph.D. Dissertation
%                        Nuno Leite
%                        september 2015
%
%/////////////////////////////////////////////////////////////////////


\documentclass[12pt, a4paper, openright, twoside]{report}

\usepackage{listings}
\usepackage{color}

\definecolor{dkgreen}{rgb}{0,0.6,0}
\definecolor{gray}{rgb}{0.5,0.5,0.5}
\definecolor{mauve}{rgb}{0.58,0,0.82}

\lstset{frame=tb,
  language=Java,
  aboveskip=3mm,
  belowskip=3mm,
  showstringspaces=false,
  columns=flexible,
  basicstyle={\small\ttfamily},
  numbers=none,
  numberstyle=\tiny\color{gray},
  keywordstyle=\color{blue},
  commentstyle=\color{dkgreen},
  stringstyle=\color{mauve},
  breaklines=true,
  breakatwhitespace=true,
  tabsize=3
}

%//to enable Code listings

%////////////
% Acronyms
% Load the package with the acronym option. Omit the dot at the end of each description
\usepackage[acronym,nomain,toc,nopostdot]{glossaries}
% Suppress acronym bold font
\renewcommand*{\glsnamefont}[1]{\textmd{#1}}

% For resetting acronyms at the beginning of each chapter
\usepackage{etoolbox}
\preto\chapter\glsresetall

%
\usepackage{glossaries-extra}
\renewcommand{\glsfirstlongdefaultfont}[1]{\emph{#1}}
\setabbreviationstyle[acronym]{long-short}
%

% Redefining indentation in the List of acronyms
\newglossarystyle{super3colleft}{%
  \renewenvironment{theglossary}%
    {\tablehead{}\tabletail{}%
     \begin{supertabular}{@{}>{\bfseries}lp{\glsdescwidth}p{\glspagelistwidth}}}%
    {\end{supertabular}}%
  \renewcommand*{\glossaryheader}{}%
  \renewcommand*{\glsgroupheading}[1]{}%
  \renewcommand*{\glossaryentryfield}[5]{%
    \glsentryitem{##1}\glstarget{##1}{##2} & ##3 & ##5\\}%
  \renewcommand*{\glossarysubentryfield}[6]{%
     &
     \glssubentryitem{##2}%
     \glstarget{##2}{\strut}##4 & ##6\\}%
  \renewcommand*{\glsgroupskip}{ & &\\}%
}



\newlength{\acronymlabelwidth}
\setlength{\acronymlabelwidth}{0.25\textwidth}
%////////////

% Chapter style
\usepackage[Lenny]{fncychap}

% Chapter minitoc
\usepackage{minitoc}

% English language
\usepackage[english]{babel}
\usepackage[T1]{fontenc}
%\usepackage[latin1]{inputenc}
\usepackage[utf8]{inputenc}

% Times New Roman typeface
%\usepackage{textcomp}
\renewcommand{\rmdefault}{ptm} % set Times as the default text font
% If AMS-LaTeX is used, it can be loaded before or after mtpro2
\usepackage{amsmath}		
%% The following loads mtpro and defines some common MTPro options 
%\usepackage[subscriptcorrection, slantedGreek, nofontinfo]{mtpro2}

% My styles
\usepackage{myStyles}


% Graphics package
%\usepackage[dvips]{graphicx}
\usepackage{graphicx}

% Other packages
\usepackage{epsfig}
\usepackage[nottoc]{tocbibind} % To include bibliography in TOC
\usepackage{latexsym}
% Fancy headers
\usepackage{fancyhdr}




%////////////// ADDED PACKAGES //////////////////
%\usepackage{natbib,hyperref} % For numerical format
\usepackage[natbibapa]{apacite} % For APA format
\usepackage{hyperref}
\usepackage{graphics} % resizebox command
\usepackage[toc,page]{appendix} % For defining appendices
\PassOptionsToPackage{hyphens}{url} % For processing correctly underscores in URL
\usepackage{subfigure}
\usepackage{calc}
%\usepackage{amssymb} % MTPro2 fonts contain all of the symbols provided by amssymb. 
% The amsthm package provides extended theorem environments
% \usepackage{amsthm}
\usepackage{amstext}
\usepackage{amsmath}
\usepackage{textcomp} % degree symbol 
\usepackage{graphicx}
%///
% Algorithm package
\usepackage[chapter]{algorithm}
%\usepackage{algorithmicx}
\usepackage{algpseudocode} % Typesetting using the algorithmicx package
% Algorithm numbering
\renewcommand{\thealgorithm}{\arabic{chapter}.\arabic{algorithm}} 
%///

\usepackage{lineno} % linenomath

% Allow break in align math environment
\allowdisplaybreaks

\usepackage{eqparbox,array}

\newcommand\LONGCOMMENT[1]{%
	\hfill$\triangleright$ \begin{minipage}[t]{15\eqboxwidth{algorithmiccomment}}#1\strut\end{minipage}%
}

\newcommand\LONGCOMMENTONE[1]{%
	\hfill$\triangleright$ \begin{minipage}[t]{14\eqboxwidth{algorithmiccomment}}#1\strut\end{minipage}%
}

\newcommand\LONGCOMMENTTWO[1]{%
	\hfill$\triangleright$ \begin{minipage}[t]{17\eqboxwidth{algorithmiccomment}}#1\strut\end{minipage}%
}

% For floating point columns
\usepackage{etoolbox} 
\usepackage[tight-spacing=true]{siunitx}
\robustify\bfseries
% For bold cells using siunitx
\newcommand{\be}[2]{%
	\multicolumn{1}{S[table-format=#1,
		mode=text,
		text-rm=\fontseries{b}\selectfont
		]}{#2}}

% For bold cells using siunitx
\newcommand{\beIt}[2]{%
	\multicolumn{1}{S[table-format=#1,
		mode=text,
		text-rm=\fontseries{b}\fontshape{it}\selectfont
		]}{#2}}

% For bold cells using siunitx
\newcommand{\beSci}[1]{%
	\multicolumn{1}{
		S[table-format =1.2e4,%    
%		S[exponent-product = \cdot,
%		round-mode = figures,
%		round-precision = 3,
		detect-weight=true, 
		detect-family=true,
		mode=text,
		text-rm=\fontseries{b}\selectfont
		]}{#1}}

%\usepackage[tight-spacing=true]{siunitx}

\usepackage{booktabs} % Professional tables
% Use in coloured tables
\usepackage{cancel}
\usepackage[pdftex, table]{xcolor}  % Coloured text etc.
%To tweak the space between columns (LaTeX will by default choose very tight columns), one can alter the column separation: \setlength{\tabcolsep}{5pt}. The default value is 6pt.
\setlength{\tabcolsep}{5pt}
%\setlength{\tabcolsep}{1.7pt}
\usepackage{todonotes}
\usepackage{multirow}
\usepackage{pdflscape}
\usepackage{enumerate}
\usepackage{stfloats} % TA plots
\usepackage{adjustbox}
\usepackage{longtable} % For long tables spanning multiple pages

% For removing underfull \hbox warning in the bibliography
\apptocmd{\sloppy}{\hbadness 10000\relax}{}{}
% Alternatively
%\usepackage{etoolbox}
%\apptocmd{\thebibliography}{\raggedright}{}{}

\usepackage{float}
\floatstyle{plaintop}
\restylefloat{table}

%\newcommandx{\info}[2][1=]{\todo[linecolor=OliveGreen,backgroundcolor=OliveGreen!25,bordercolor=OliveGreen,#1]{#2}}

\newif\ifboldnumber
\newcommand{\boldnext}{\global\boldnumbertrue}

% Default definition is \footnotesize#1:
\algrenewcommand\alglinenumber[1]{%
	\footnotesize\ifboldnumber\bfseries\fi\global\boldnumberfalse#1:}

\usepackage{verbatimbox}

\newcolumntype{L}[1]{>{\raggedright\let\newline\\\arraybackslash\hspace{0pt}}m{#1}}


\usepackage{doi}
\renewcommand\doitext{} % Remove duplicate "doi:" prefix
% For placing figures and tables in the next page
\usepackage{afterpage}
\usepackage{flafter}
% siunitx EN language floating point separator
\sisetup{group-separator = {,}}

% For reducing \hbox warnings
\sloppy

%////////////////////////////////////////////////


\usepackage{setspace}
\onehalfspacing % or \doublespacing

% Line spacing
%\linespread{1.25} %// LAST
%\linespread{1.35}
%\linespread{1.4}

% Page dimensions - equivalent to 15 cm x 23 cm
\usepackage[left=3.5cm,right=2.5cm,top=3cm,bottom=3.7cm]{geometry}

%\usepackage[headheight=14pt, text={15cm,23cm}, twoside, bindingoffset=1cm]{geometry}

%\usepackage[headheight=14pt, text={15cm,23cm}, top=2cm, left=2.5cm, right=2.5cm, bottom=2.5cm, twoside, bindingoffset=1cm, includefoot, includehead]{geometry}

% Page dimensions
%\pagestyle{plain}
%\setlength{\oddsidemargin}{0.5cm} 
%\setlength{\evensidemargin}{-0.5cm}
%\setlength{\topmargin}{-1cm} 
%\setlength{\textheight}{23cm}
%\setlength{\textwidth}{16cm}
%



%-
% //////// Changing default figure captions ///////
%-
%\setlength{\abovecaptionskip}{10pt}
\setlength{\belowcaptionskip}{5px}


%\makeatletter % Allow the use of @ in command names
%\long\def\@makecaption#1#2{%
%\vskip\abovecaptionskip
%\sbox\@tempboxa{ \textbf{#1} \hspace{0.5em} #2}%
%\ifdim \wd\@tempboxa >\hsize
%{ \textbf{#1} \hspace{0.5em}  #2\par}
%\else
%\hbox to\hsize{\hfil\box\@tempboxa\hfil}%
%\fi
%\vskip\belowcaptionskip}
%\makeatother % Cancel the effect of \makeatletter 
%% ////////////////////////////
%
%% //// Use 'Computer Modern' typewriter font, not 'Lucida Typewriter' ////
%\renewcommand{\ttdefault}{cmtt}%
%% //// Estilo adoptado para trocos de codigo no texto ////
%\newcommand{\code}[1]{{\tt #1}}



% ///// Fancy headers ////////
\usepackage{fancyhdr}
\pagestyle{fancy}
\fancyhf{}

\renewcommand{\chaptermark}[1]{%
  \markboth{\chaptername \ \thechapter \hspace{0.5em} \emph{#1}}{}}

\renewcommand{\sectionmark}[1]{%
  \markright{\thesection \hspace{0.5em} #1}{}}
     
\renewcommand{\headrulewidth}{0pt}

\fancypagestyle{plain}{%
\fancyhf{} % clear all header and footer fields
\fancyfoot[C]{\small\thepage} % except the center
\renewcommand{\headrulewidth}{0pt}
\renewcommand{\footrulewidth}{0pt}}
% /////////////////////////////////


% Load acronym definitions
\loadglsentries[\acronymtype]{Acronyms/Acronyms}

% Generate the glossary
\makeglossaries


\usepackage{silence}

\WarningFilter{minitoc(hints)}{W0023}
\WarningFilter{minitoc(hints)}{W0028}
\WarningFilter{minitoc(hints)}{W0030}

\WarningFilter{blindtext}{} % this takes care of the `blindtext` messages




%//////////////////////////
%
% Main matter
%
%//////////////////////////
\begin{document}

% Environment for displaying examples
%\theoremstyle{definition}
\newtheorem{exmp}{Example}[section]


% Commands
\renewcommand\bibname{References}


% Document cover
\thispagestyle{empty}

\begin{figure}[ht!]
	%\vspace*{-4cm}
	\centering
	\scalebox{0.6}{\includegraphics{FrontCover/Logo_ISEL}}
\end{figure}


\centerline{\Large\textbf{INSTITUTO POLIT\'{E}CNICO DE LISBOA}}
\bigskip
%\bigskip
\centerline{\Large\textbf{INSTITUTO SUPERIOR DE ENGENHARIA DE LISBOA}}


\bigskip
\bigskip

\title{{\large Licenciatura em Engenharia de Inform\'{a}tica e de Computadores}}

\bigskip
\bigskip

\begin{center}

 
{\LARGE \textbf{Event Organizer}} 

\bigskip
\bigskip
\bigskip
\bigskip
\bigskip



{\Large \textbf{\vspace*{1em}Marco Batista n.\textdegree $\ $42125, e-mail: \texttt{a42125@alunos.isel.pt}\\ Diogo Martins n.\textdegree $\ $42393, e-mail: \texttt{a42393@alunos.isel.pt}}}

\bigskip
\bigskip
\bigskip
\bigskip
\bigskip

\begin{table}[h]
	\begin{tabular}{@{}l@{\hspace{1.5cm}}l} \vspace{1.5mm}
		{\Large \textbf{Supervisor:}}   & {\Large Nuno Leite} \\% 
	\end{tabular}
\end{table}


\bigskip
\bigskip
\bigskip


{\Large Projeto e Semin\'{a}rio}


\bigskip
\bigskip
\bigskip
\bigskip
\bigskip

{\Large Report}

\bigskip
\bigskip		
\bigskip


\bigskip
\bigskip


{\large \textbf{September 2019}}


\end{center}
 

\clearpage
\thispagestyle{empty}
\cleardoublepage



% Begin with roman numbering at page 3 (provisory document)
\pagenumbering{roman} \setcounter{page}{3}
%\pagenumbering{roman} \setcounter{page}{5}



% Fancy Header configuration

% Formato da pagina esquerda (par): <Pagina><Capitulo nr Nome>
\fancyhead[LE]{\small\thepage\hspace{3em}\nouppercase{\small\leftmark}}
\fancyhead[RE,LO]{}
% Formato da pagina direita (impar): <Numero da Seccao> <Nome da seccao> <Pagina>
\fancyhead[RO]{\nouppercase{\small\rightmark}\hspace{3em}\small\thepage}

\setlength{\headheight}{13.60pt}


% //////// ORIENTACAO DA TESE //////////////
%\typeout{}
%\include{../Orientation/Orientation}
%
%\newpage \mbox{}    % Blank page.
% //////////////////////////////////////////


% English Abstract
\typeout{Abstract}

\prefacesection{Abstract}



\vspace{3em}

\noindent

There is a large library of platforms and apps that allow one to invite people to an event. These platforms, however, have some downsides, for example, splitting expenses of an event is hard to do on those platforms, they require new users to make an account and activating it, are not available on multiple platforms, among other issues. The goal of this project is to develop a new platform that tackles these disadvantages. This way the group implemented a new solution using modern platforms (such as ASP.NET CORE and Xamarin) that can be used by a large number of people on modern devices and can later be expanded to new environments that might be created. The group was presented with some challenges that had to be overcome, like the authentication without account creation using multiple providers, the notifications mechanism, or the code sharing between platforms. The result is a new application supported by a Web API without the need to create a new account on another platform, using the already existing accounts on other services like Facebook and Microsoft’s Outlook, for example.


\vspace*{1em}

\textbf{\large{Keywords:}} Events, Item, Task, User, Invitation.


%\begin{itemize}
%	
%	
%	\item Events - Something that takes place (in this case social events);
%	\item Item - Associated with events, these are the objects of an event;
%	\item Task - An action that needs to be done by a participant in an Event;
%	\item User - Person that uses the platform, normally comprised of a Name, Email and Phone Number;
%	\item Invite - Asking for someone to go to an Event.
%	
%\end{itemize}

%
%
%\cleardoublepage
%
%
%\prefacesection{Acronyms}
%
%
%
%\vspace{3em}
%
%\noindent
%
%%\textbf{\large{Acronyms:}} 
%
%\acrshort{orm} stands for \acrfull{orm}.
%
%\acrshort{crud} stands for \acrfull{crud}.
%
%\acrshort{ide} stands for \acrfull{ide}.






%\newpage \mbox{}    % Blank page.


% Portuguese Abstract
%\typeout{Resumo}
%



\prefacesection{Resumo}





\vspace{3em}
%\vfill{}

\noindent
\textbf{\large{Palavras-chave:}} Keyword 1, Keyword 1, ... .

%\newpage \mbox{}    % Blank page.


% Acknowlegdments
%\typeout{Acknowledgements}
%

\prefacesection{Acknowledgements}


I would like to thank to the people and the institutions directly involved in this work.

To ....
Finally, I would like to thank my family. ....




%\newpage \mbox{}    % Blank page.
%
%
%% //////// Dedication //////////////
%\typeout{}
%


\vspace*{8em}


\hspace*{\fill} {\large \textit{To ...}}

\vspace*{4em}

\hspace*{\fill} {\large \textit{To ...}}

%\newpage \mbox{}    % Blank page.
% //////////////////////////////////////////


%% //////// List of Acronyms and Abbreviations //////////////
\printglossary[type=\acronymtype,nonumberlist=true]
%{\typeout{List of Acronyms and Abbreviations} 
%


%%/////////////////////////////////////////////////
% Acronym definitions
% <key> <acronym> <description>
\newacronym{crud}{CRUD}{Create, Read, Update and Delete}
\newacronym{orm}{ORM}{Object-Relational Mapping}
\newacronym{ide}{IDE}{Integrated development environment}
\newacronym{dal}{DAL}{Data Access Layer}
\newacronym{api}{API}{Application Programming Interface}

%%/////////////////////////////////////////////////







}
%
%\newpage \mbox{}    % Blank page.
%% //////////////////////////////////////////

% List of Acronyms with no page numbers
%\printglossary[title=List of Acronyms,toctitle=List of Acronyms,nonumberlist=true]

%
%% //////// Notational Conventions //////////////
%{\typeout{Notational Conventions}
%%\vspace{1cm}
%\begin{flushleft} 
%\fontsize{16}{24}\textbf{Lista de Acrónimos} 
%\end{flushleft} 
%\vspace{1cm}
\prefacesection{List of Symbols and Notation}
\vspace{-2cm}
The most frequent symbols and notation are as follows.
\hrule
\begin{table}[h]
\begin{tabular}{@{}l@{\hspace{.5cm}}l} \\
$c,$	 		& number of different classes in a dataset \vspace{-2mm} \\
$d,$	 		& dimensionality or number of features in a dataset \vspace{-2mm} \\
$L,$ 			& cumulative threshold for feature selection \vspace{-2mm} \\
$m,$	 		& number of selected features \vspace{-2mm} \\
$M_S,$    & maximum allowed similarity \vspace{-2mm} \\  
$n,$	    & number of instances in a dataset \vspace{-2mm} \\
$q,$      & maximum number of bits to discretize a given feature \vspace{-2mm} \\
$s,$      & number of sampled instances from a dataset \vspace{-2mm} \\ 
${\bf x}_i,$  & the $i$-th feature vector $\in \mathbf{R}^d$ \vspace{-2mm} \\ 
$X_i,$ & feature $i$ with $i \in \{1,\ldots,d\}$ \vspace{1cm} \\ 

$\eta,$   & percentage of features to keep in the DDF and the DDW methods \vspace{-2mm} \\
$\ell_0,$ & $\ell_0$ norm for a vector \vspace{-2mm} \\
$\Delta,$  & maximum allowed distortion \vspace{1cm} \\

$\mathcal{D},$ & dataset \vspace{-2mm} \\
$\mathcal{X, Y},$   & random variable \vspace{1cm} \\

$H(.),$ & entropy of a random variable \vspace{-2mm} \\
$H(.|.),$ & conditional entropy of one random variable given another \vspace{-2mm} \\
$I(.),$  & self-information of a specific random variable outcome \vspace{-2mm} \\
$I(.;.),$  & mutual-information between a pair of random variables \vspace{-2mm} \\

\end{tabular}
\end{table}



} % Blank page.
%% //////////////////////////////////////////
%
%
%% //////// List of Symbolss //////////////
%{\typeout{List of Symbols}
%\include{Other/ListOfSymbols}} % Blank page.
%% //////////////////////////////////////////
%



%///////////////////////////////////////////////
%
% Contents
% List of Figures
% List of Tables
% List of Algorithms
%
%///////////////////////////////////////////////
\dominitoc
\tableofcontents 
%\adjustmtc
\listoffigures
%\adjustmtc
%\listoftables   
%\listofalgorithms \addcontentsline{toc}{chapter}{List of Algorithms} 

\clearpage

% Arabic numbering
\pagenumbering{arabic}
 

\setcounter{mtc}{3} % First page containing the minitoc


%////////////////////////////////////////////////////////////////////////
%   Chapter 1
%
%   Introduction
%////////////////////////////////////////////////////////////////////////



\chapter{Introduction} 
\label{ch:Chapter1}
\vfill \newpage
\noindent

PostChat is an innovative Android application that aims to bring back the charm of postcards and bridge the gap between the older and younger generations through the use of modern technology. The app serves as a client-server platform that enables users to create, personalize, and send digital postcards to their loved ones, fostering meaningful connections in today's digital age.
Figure~\ref{fig:OV} illustrates PostChat's use case.

\begin{figure}[!ht]
	\centering
	\includegraphics[width=0.75\textwidth]{./Chapter1/Figures/Overview}
	\caption{Overview}
	\label{fig:OV}
\end{figure}

In today's fast-paced world, people have become increasingly reliant on technology to communicate with their friends and family. However, the traditional art of sending postcards has lost its appeal over the years, particularly among the younger generation. With PostChat, we strive to revive this age-old practice and make it relevant again, while also making it accessible and convenient to use for everyone.

The PostChat application operates on a client-server architecture, where the Android app serves as the client and a web API functions as the server. This design allows for seamless communication between users and the server, enabling the creation and transmission of digital postcards with ease.

The PostChat app is meticulously designed to cater to the needs of all age groups, with a user-friendly interface that is easy to navigate. Users have access to a diverse range of postcard templates, allowing them to personalize their messages and express their creativity. Additionally, the app integrates advanced features, such as AI tools tailored for the visually impaired, ensuring an inclusive and empowering experience for all users.

Furthermore, PostChat goes beyond the confines of traditional postcards, as it facilitates real-time communication and community building. Users can create chat groups within the app, enabling them to engage in conversations and share postcards with multiple recipients simultaneously. This feature encourages meaningful interactions and strengthens connections among friends and family.

This report aims to delve deeper into the workings of the PostChat application, analyzing its features, design, usability, and the integration between the Android app and the web API server. The report is organized like this:
\begin{itemize}
	\item Requirements - What has to be done in order to implement this project;
	\item Web Api - The server side application;
	\item HTR Model - Our approach to a HTR model it's limitations and challenges;
	\item Client - The Android client application;	
\end{itemize}



%////////////////////////////////////////////////////////////////////////
%   Chapter 2 
%////////////////////////////////////////////////////////////////////////

\chapter{State of the art} 
\label{ch:Chapter2}
\vfill \minitoc \newpage

It's possible to identify some applications that include an event planning function: \textit{\cite{facebook}}, \textit{Doodle}, \textit{\cite{superplanner}}, and \textit{\cite{asana}}. The first two listed, \textit{Facebook} and \textit{Doodle}, both support the creation of events, but they focus on scheduling while also allowing for some basic functionalities like invitation of people to the event. 

\textit{Super Planner} allows for more complex event management, providing calculators for venue capacity, food/catering and staffing, as well as others for managing a budget limit. 

\textit{Asana} can be used to assign tasks to people and manage the work that needs to be done. However, none of these have a focus on collaboration between people focused on an event. While \textit{Asana} has the capability of assigning tasks between people, it serves more as a work management tool to organize teams in business projects.

\textit{Super Planner} has good tools to manage an event, but lacks collaboration between people and invitation of people to the event is not possible. 

\textit{Facebook} and Doodle are great tools for basic event planning, but do not offer much more than setting a location, date and giving a description of the event. It is important to mention that \textit{Facebook} does offer some tools that could prove useful, such as posting a message in the event page, but that are not specifically made for this purpose and end up being too cumbersome.

The current state of the art now counts with a really similar application to what we are proposing, an application called \textit{\cite{eventbrite}}.
This application has a really similar concept but in a different way. First of all it implements a ticket system. Defining an event requires you to make a ticket price, which is not ideal since the group wants the event's expenses to be modifiable throughout the event, implementing a voting system for it if wanted. There are others that are more focused on team management, for projects, like \textit{\cite{cvent}} and \textit{\cite{xingevents}} and then there are ones like \textit{EventBrite} that implement a ticket system like \textit{Ticket Taylor}. 

%Also one has to create an account to use this platform, which is not ideal because it's limiting of a big user base as people are normally reluctant to create accounts on platforms as there are too many is the current days.

Also one has to create an account to use this platform, which is not ideal because people are normally reluctant to create accounts on platforms as there are too many in the current days. 


As far as the group knows there are none that implement a division of tasks, expenses and playlist creation that have a focus on collaboration of people, as well as multimedia functionality. These features make sense together in an event planning app as a big volume of events contain all of these aspects in common.


%////////////////////////////////////////////////////////////////////////
%   Chapter 3
%
%////////////////////////////////////////////////////////////////////////
\chapter{Requirements}
\label{ch:Chapter3}
\vfill \minitoc \newpage

\section{Functional Requirements}

\subsection{Mandatory Requirements}

The mandatory functional requirements are the following:
\begin{itemize}
	\item Creation of an Event;
	
	\item Definition of the location of the event using a maps platform (for example, \textit{Google Maps});
	
	\item Guests invitations;
	
	\item Guest task managing;
	
	\item Event expense managing (including the voting system);
	
	\item \textit{Playlist} creation.
\end{itemize}


\subsection{Optional Requirements}

\begin{itemize}
	\item Ease of payment through services like \textit{PayPal} or \textit{MbWay};
	\item Ease of expense managing by adding products through their barcode;
	\item Deployment of the APIs on the cloud (for example: \textit{Azure} or \textit{Google Cloud})
\end{itemize}

\section{Functionallity}

The platform allows the user to signup. Figure~\ref{fig:UserSignupScreenShot} illustrates the \textit{User Signup} operation.

\begin{figure}[!ht]
	\centering
	\includegraphics[width=0.60\textwidth,height=0.50\textheight]{./Chapter3/Figures/ClientAppScreenShots/Signup01}
	\caption{User Signup Screen}
	\label{fig:UserSignupScreenShot}
\end{figure}

\newpage
Users are redirected to the facebook login page to signup. It's possible to use \textit{Facebook} as an identity provider.
Once a user is signed in, it has the ability to create an event, giving its Title and Description, also the start and end dates. After the event is created, it can add its initial expenses/items so the invited users know if they need to pay for anything in advance, and how much.

Figure~\ref{fig:CreateEvent} illustrates the \textit{Event creation} operation.

\begin{figure}[!ht]
	\centering
	\includegraphics[width=0.25\textwidth,height=0.50\textheight]{./Chapter3/Figures/Flowcharts/CreateEventFlowChart}
	\caption{Create Event Flowchart}
	\label{fig:CreateEvent}
\end{figure}

\newpage

Figure~\ref{fig:CreateEventScreenShot} illustrates the \textit{Event creation} screen.

\begin{figure}[!ht]
	\centering
	\includegraphics[width=0.60\textwidth,height=0.50\textheight]{./Chapter3/Figures/ClientAppScreenShots/CreateEvent04}
	\caption{Event Creation Screen}
	\label{fig:CreateEventScreenShot}
\end{figure}

\newpage

To add an item the user has to choose an event and then specify a title, description, price (if the item has a price) and it's type (types can be: \textit{Food}, \textit{Drink}, \textit{Decoration}, \textit{Taxes} and \textit{Others}).

Figure~\ref{fig:ItemCreationScreenShot} illustrates Item Creation.

\begin{figure}[!ht]
	\centering
	\includegraphics[width=0.60\textwidth,height=0.50\textheight]{./Chapter3/Figures/ClientAppScreenShots/AddItem_08}
	\caption{Item Creation Screen}
	\label{fig:ItemCreationScreenShot}
\end{figure}

\newpage

When items are created in the event a guest can see the event extract. This screen will present to the guest who is paying for what items and how much.

By selecting an item, inputting how much one has payed for it and clicking "Change Payed Value" an entry will be added to the event extract so other guests can see how much the others payed and how much it has to pay also.

Figure~\ref{fig:EventExtractScreenshot} illustrates Event Extract.

\begin{figure}[!ht]
	\centering
	\includegraphics[width=0.60\textwidth,height=0.50\textheight]{./Chapter3/Figures/ClientAppScreenShots/Extract_01}
	\caption{Item Creation Screen}
	\label{fig:EventExtractScreenshot}
\end{figure}



\newpage

After the user has at least one event created it can see all the events that it created until now and when the user chooses an event, it can list and edit items aswell as invite users:

Figure~\ref{fig:EditEvent} illustrates the Event edit.

\begin{figure}[!ht]
	\centering
	\includegraphics[width=0.25\textwidth,height=0.50\textheight]{./Chapter3/Figures/Flowcharts/EditEventFlowChart}
	\caption{Edit Event Flowchart}
	\label{fig:EditEvent}
\end{figure}

\newpage

Figure~\ref{fig:EditEventScreenShot} illustrates the Event edit.

\begin{figure}[!ht]
	\centering
	\includegraphics[width=0.60\textwidth,height=0.50\textheight]{./Chapter3/Figures/ClientAppScreenShots/Event13}
	\caption{Edit Event Screen}
	\label{fig:EditEventScreenShot}
\end{figure}

\newpage

It's also possible to add Tasks to users. Tasks are something that have to be completed until a certain expiration date. Tasks are comprised of a name, description and expiration date. They are also associated with a user so when a task is assigned to one they receive a notification.

Figure~\ref{fig:AddTaskToEvent} illustrates adding a Task to an Event.

\begin{figure}[!ht]
	\centering
	\includegraphics[width=0.60\textwidth,height=0.50\textheight]{./Chapter3/Figures/ClientAppScreenShots/AddTask_01}
	\caption{Edit Event Screen}
	\label{fig:AddTaskToEvent}
\end{figure}

\section{Events Web API}

The design of the API follows a REST approach over HTTP. This was decided so that the client/server side of the project can be \textit{scalable} and \textit{loosely coupled}, meaning that in the future, more clients for this API can be developed without the need to change anything to accommodate them.

%In order to try to decouple the server side from the client even more a \textit{Hypermedia As The Engine of Application State}, HATEOAS for short, component is also being considered as this would allow for the clients of the API to navigate it by using the \textit{hypermedia} links obtained in the various responses.%

In the current version, the Events API has the endpoints presented in Figure~\ref{fig:Endpoints}.


\begin{figure}[!ht]
	\centering
	\includegraphics[width=1\textwidth,height=.45\textheight]{./Chapter3/Figures/EventsAPI_Endpoints}
	\caption{Events API endpoints in use by client application}
	\label{fig:Endpoints}
\end{figure}


The API is using Entity Framework, ADO.NET as the technologies to access data of the groups PostgreSQL database.

\newpage



\subsubsection{Implementation of the API}

\paragraph{Basic API structure}

The API, as mentioned earlier, is based on .NET Core. The controllers are called by the Http requests and the services that they use are initialized by the Dependency Injection engine that the framework provides. The dependencies are all configured on the Startup.cs file of the project.
The services contain all the business logic of the API and when they need to reach data to operate upon these call the methods exposed by the repository wrapper implemented in the API.

The repository wrapper allows for easy access to data repositories, being similar to a chest of items where one can get all of the data they need centralized in one class.

All of the repositories extend RepositoryBase<T> which exposes all the \gls{crud} operations while allowing for deferred execution when querying the database. Deferred execution allows for creating the query and only executing it against the database when a terminal method is called (for example .ToList()). This allows for a reduction of the queries ran on the database meaning it's possible to get all the data from a custom query only on one travel to the database server.


\paragraph{Logging}

The API allows for logging by using a group implemented Logging Provider and adding it to the Startup on configuration using \textit{ILoggerFactory}. The logging level is controlled by the appsettings file of the project where there are 7 levels:

\begin{itemize}
	\item Trace
	\item Debug
	\item Information
	\item Warning
	\item Error
	\item Critical
	\item None
\end{itemize}

Logging is called (or not) automatically by the platform according to the level defined in the settings.

In case of an occurrence of an unhandled exception there is an ErrorController than has an Error method that is a generic exception handler. This exception handler returns a 500 Http Status Code and before it returns to the user it sends an email to the groups application email box with a Message and StackTrace of the Exception. If the logging level is defined for every one except None it also gets logged to the logging database.

The logging database DAL is implemented using ADO.NET because of the fast execution of this technology while accessing a database. The tradeoff here is that this performance, which allows for the logging to be fast and less impact-full of the API performance overall, comes at a cost of the flexibility and code maintenance that Entity Framework provides.

\paragraph{Email Sending}

The email sending is used as the main notification system. It is used in the following situations:
\begin{itemize}
	\item When a user is registered an email is sent to its mailbox
	\item Upon the creation of an item or a task (sending the task to the person whom that task was assigned too)
	\item Sent to the group's mailbox when an Exception happens.
\end{itemize}

for when a new user is registered (sending a welcome email to the user mailbox) and on an unhandled exception.
The methods defined on this service are asynchronous methods so they run in the background as SMTP communication usually isn't very fast so this way the API overall performance is not impacted a lot in a negative way. The class the group is using to send emails is \textit{SmtpClient} and it's configuration is given by the appsettings file. The email provider the group is using is \textit{Outlook}.


\paragraph{ModelState Validation}

This API takes advantage of the technology of model validation present in the framework. This validation technology goes by the name of ModelState Validation. This technique consists on decorating the models used by the client application with Data Annotations.
These Data Annotations are placed in the properties of the model as Attributes. These can define if a property is required, what is the maximum length of a property (for example, in case of a string), the value range, etc…
The client application sends the model and when the request reaches the web API controller, one can check if the model sent by the client app is valid, by calling ModelState.IsValid, before sending the request to the other layers of the API (service, data, etc…).
If ModelState.IsValid returns true, then the model obeys to all validations described in the data annotations. However, if false, the request isn’t sent to the other layers, and on the response a body containing the validation errors will be sent (accompanied by a HTTP 400 – Bad Request – status code). These validation errors will help the client understand what is wrong with it’s request so they can correct it.

\section{Music Web Api}

The \textit{Music Web API} is a set of definitions that aim to unify music providers by having them expose common functionalities through a single \textit{Web API definition}.

\subsection{Structure}
The \textit{Music Web API} repository contains three main projects that are:

\begin{itemize}
	\item MusicWebApi – This project contains the interfaces that need to be implemented in order to follow the Music Web Api definition.
	
	\item MusicWebApi.Client – .NET Standard project that eases the use of Music Web Api Services by abstracting HTTP requests in simple method calls.
	
	\item MusicWebApi.Models – Another .NET Standard project shared by the first two containing the model objects for the API requests and responses.
\end{itemize}

In addition to these projects there are also projects that contain the implementations of the \textit{Music Web API}, such as \textit{SpotifyMusicWebApi}, an implementation for the popular Spotify music service.

\subsection{Definition}
As previously mentioned, the \textit{Music Web API} is an API definition for services that wish to expose the functionality of music providers, existing or native, in a uniform and standard interface.

Music applications can make use of this if there is a need to have multiple music providers without having to create multiple user interfaces or API access layer implementations. This can prove useful in both the maintainability of the code by keeping it uncluttered and centralized (only one API access service is needed) and in keeping the application easily expandable simply by adding to the music services available, without the need of big code changes.

The following diagram exemplifies the functionality of the \textit{Music Web API} ecosystem described earlier:

\begin{figure}[!ht]
	\centering
	\includegraphics[width=1\textwidth,height=.45\textheight]{./Chapter3/Figures/MusicWebApi/MusicApiDiagram.png}
	\caption{Diagram of the Music Web API functionality}
	\label{fig:MusicApiDiagram}
\end{figure}

\begin{figure}[!ht]
	\centering
	\includegraphics[width=1\textwidth,height=.20\textheight]{./Chapter3/Figures/MusicWebApi/PlaylistControllerTable.png}
	\caption{Playlist Controller}
	\label{fig:PlaylistControllerTable}
\end{figure}

\begin{figure}[!ht]
	\centering
	\includegraphics[width=1\textwidth,height=.025\textheight]{./Chapter3/Figures/MusicWebApi/TrackControllerTable.png}
	\caption{Track Controller}
	\label{fig:PlaylistControllerTable}
\end{figure}

The decision to keep the \textit{create playlist} and \textit{add tracks to playlist} endpoints working with the POST method instead of PUT was made because the \textit{Music Web API} definition cannot guarantee that, in the case where implementations use third party providers such as Spotify, the creation and addition of tracks is idempotent.

Although it makes the API less fault tolerant overall, it is a necessary trade off to make it broader by supporting implementations that use third party services.

\subsection{Authentication and authorization}
All music services that follow the definition \textit{Music Web API} must use the industry-standard protocol, OAuth 2.0, for authorization.

As such among the three controller interfaces in the MusicWebApi project an AuthController can be found. This controller has only two methods, one that begins the authentication flow for the user and another that serves as a redirect URI that must resolve the code and return a \textit{TokenResponse} object, which represents the standard OAuth 2.0 response, in the HTTP body.

\subsection{Client}
The client project allows for safer calls to a Music Web Api service by having methods that receive and return the exact model objects and build the appropriate HTTP requests. This makes applications that use these services cleaner and faster to develop, only needing to import the nugget package for this client and call the methods wanted.

\section{Client Application}

The initial plan for the client application was \textit{"a mobile application, using the \textit{Android} platform. The development environment will be .NET with the \textit{Xamarin} SDK."}. The group decided to use the Xamarin Forms variation of the environment since it has two advantages when compared to Xamarin Native:

\begin{itemize}
	\item The ability to share similar UI between two different platforms (in addition to application code);
	
	\item Easier maintenance since even more code is shared.
\end{itemize}

The group decided it was best to sacrifice application size (as this is the biggest con to Xamarin Forms) for better code organization and overall looks.

This decision also made it so there are now two client applications instead of just one. Right now it's possible to run the event organizer client application on both Android and UWP/Windows.

The platform supports the current operations:

\begin{itemize}
	\item Events - Create events;
	
	\item Items/Expenses - Addition of Expenses to an event;
	
	\item Tasks - Add tasks;
	
	\item Events - Cancel an event;
	
	\item Register Users - Registering users using authentication providers (\textit{Facebook})
	
	\item Invites - Invite users to participate in an event;
	
	\item Add Payment - Associate payment with a certain item and see the current extract of the event
\end{itemize}




\subsubsection{Implementation of the client app}

\paragraph{Layout and basic application structure}

This application as stated earlier is implemented using Xamarin Forms. This toolkit serves as our view engine for the application and suggests ways of organizing the client application so code can be reused between platforms.

The application has Content Pages that are the Views of the app. These are written using \textit{Extensible Application Markup Language} (XAML) which is a declarative markup language that allows to create the user interfaces for the application. These are written as files with the ".xaml" extension.
\newline
The XAML content pages have a .cs file associated with them that allow C\# code to be written to them by subscribing to events that are exposed by the Xamarin Forms platform.
This is simillar to ASP.NET's \textit{Web Forms}.

For example, when navigating between views there is a concept of "Push" and "Pop". Push meaning going forward and Pop going backwards in the page navigation.

When one is "Pushing" to a new view that view needs to be instantiated with the \textit{new} operator, executing the views ctor. The constructor shouldn't have code that takes time to complete in its body, only layout changes or event declaration should happen here, for example.

If one has to, for example, call the API when navigating to a new view one should subscribe to the \textit{OnAppearing} method which is called when the view is rendered on the target device.

\paragraph{Local Notifications}

Local notifications are handled with the help of a Xamarin Plugin called \textit{Xam.Plugins.Notifier}. This plugin allows for cross-platform local notification handling by exposing three methods to the programmer.
The group decided to use this plugin because of it's simplicity and the easiness of allowing the same notifications working for all platforms, in this case, UWP and Android. 
However there are some limitations with this plugin:



\begin{itemize}
	\item There is no way to get a list of the current notifications of the App;
	
	\item The latest (stable) version of the Nuget package present on the GitHub repo, as of 2019-05-18 is not working with scheduled notifications on UWP (meaning the group had to search the github repo of this plugin and on pull request \#43 https://github.com/edsnider/localnotificationsplugin/pull/43 this was fixed).
	
	
\end{itemize}

\newpage

The way the group handled the cancelling of scheduled notifications was to save the notifications according to an Id that could be easily retrieved when canceling, by giving the object that caused the notification to happen and an id:

These are the method signatures that handle this problem:
\begin{itemize}
	\item public void AddNotificationCountByObjectAndId<T>(T obj, int id, int count)
	where T : class - for adding notifications
	
	\item public void CancelNotificationByObjectAndId<T>(T obj, int id)
	where T : class - for cancelling notifications
	
	
\end{itemize}


For example if one wanted to add two notifications for an Event one would pass the EventModel object and its Id. This would save a configuration on the device after scheduling the notifications and if you wanted to cancel all of the notifications you would call CancelNotificationByObjectAndId with the same parameters that were supplied to an earlier call to AddNotificationCountByObjectAndId.

%Note that with this implementation it's not possible to cancel a specific notification with this implementation. 

Note that with this implementation it's not possible to cancel a specific notification. In case one wanted to cancel one it can be done by cancelling all notifications for a specific object and then creating new ones.

The handling of saving the configuration is done with the help of Xam.Plugins.Settings as used in other parts of this projects client application.

\paragraph{API Request Handling}

The application requests the groups API using a Request Service that is designed using \textit{HttpClient} and a class that the group implemented called NetworkService. This NetworkService class in conjunction with \textit{Polly} exposes a "Retry" method that allows for an arbitrary number of attempts of requesting the API in case a network communication fails, while allowing the group to execute a function on each attempt (for example, this can be used on a case where the user could be warned about the operation taking longer than usual). However the Request Service only uses the Network Service on GET methods because of the retry being a problem on non idempotent operations, which could cause inconsistencies to the application state.

\paragraph{Permission Handlers}

The maps and contact providers are implemented with permission handling in mind. Users might be prompted to grant permission to a functionality in the application, and the application should behave accordingly to the granting or denying of the permission.
On Android permission handling is different from other platforms, as on the user decision it calls a method on the MainActivity class of the application, meaning that the programmer will lose context from the page that requested the permission for a specific functionality.
To solve this the providers that require permissions have interfaces that expose callback setters, so immediately after the user grants or denies permission, there is behavior attached to its choice (for example, going back in the navigation stack when a user denies access to something).
Callback setters are useful and flexible because they separate the concerns to the provider. For example, after the user grants permissions to the contact list, it should be provided on the screen. For this functionality you set a callback to be executed after the granting of the permission by the user.
On UWP this isn’t needed, as the permission checking is not based on calling a method on another class. The UWP API normally exposes methods that return the permission status on the fly and then ask the user if the application does not have permission.
As a side note there is a Xamarin plugin for handling permissions called Plugin.Permissions. However, this plugin, by analyzing its source code, is somewhat heavy on the performance side, as to handle the permission architecture of android there is a need to use locking mechanisms to synchronize the permission request and its subsequent status.


\paragraph{Location/Maps}

The definition of the location of the event is supported by a plugin by the name of Xamarin.Forms.Maps. This plugin allows for the cross-platform integration of maps in a mobile application, providing a simple to use interface that is common to every platform.
The plugin is cross-platform, however, for each platform there is a different map provider:

\begin{itemize}
	\item Android – Google Maps map provider
	\item UWP – Bing map provider
	\item iOS – Apple maps map provider
\end{itemize}


However, when using a map provider, the application needs to ask the user for permission (mainly for location). This was handled by creating a wrapper around each provider and supplying instances to the cross-platform project of the application via dependency injection.
These wrappers handle permission handling as well as every map operation possible that the application needs to support (moving the map to a specific location, showing a dialog when the user doesn’t give permission for map usage, adding pins on the map, etc…).
The group implemented the map feature for the Android and UWP platforms.
As seen in the contacts provider, handling permissions on Android is particularly hard compared to other platforms. The Android runtime calls a specific method in the main activity to handle permissions, hence the need to pass page components to the maps provider (to call methods and implement functionality after the user grants permission.)

If the request for permission is denied, then the user is returned to the previous page. However, on the UWP application the settings page for permissions of the app is opened automatically, so the user can grant permission for map usage.

The search component of the map functionality of the application is provided by the OpenStreetMap API. The group decided to use this API as it is free, easy to use (there’s not even a need for requesting an API key). The only restrictions there are for this API is that you shouldn’t abuse requesting it and it is obligatory to identify your application by providing a custom user-agent in the HTTP headers. The user searches an address, the results are presented, and after the user chooses one from the list, the coordinates are passed into the map provider (giving the opportunity for the user to save the event location).

\paragraph{Contacts}
When one is inviting a user to an event a list of contacts present in the invitee's device is shown. The list of contacts is brought with the help of contact providers and a Xamarin plugin called Xamarin.Forms.Contacts.
The Xamarin plugin is straightforward to use. This plugin provides a service and there is only need to call a method on that service. However this plugin does not handle permissions and it does not work on UWP. So the group implemented a "hybrid" solution with contact providers.
Contact providers offer an interface to implement contact functionality on multiple platforms.
This contact provider allows one to supply the list view that is going to hold the contacts to an instance of itself. This is especially useful for handling Android permissions.


\newpage
\subsection{Database}

The application and logging databases are being supported by PostgreSQL. The group choose this engine over others because this one is free and open-source, it also supports a wide range of operating systems and is usually faster to install and setup.

%This is the current data model:

Figure~\ref{fig:EAModel} illustrates the data model.

\begin{figure}[!ht]
	\centering
	\includegraphics[width=0.75\textwidth]{./Chapter3/Figures/EventDatabase.png}
	\caption{Data Model}
	\label{fig:EAModel}
\end{figure}


\begin{itemize}
	\item Event - Represents the Events, has a Title, Description and Start/End dates of the event;
	
	\item Item - Represents the Items (or expenses) of the Event. Has a Name, Description, Price, State and it can have various types;
	
	\item Task - Represents the Tasks of the Event assigned to participants. Has a Name, Description, Expiration Date, State and references an event and a participant;
	
	\item BufferedItem - It is the same as the Items except it represents only the items that have been successfully approved by all members of the event (note the VoteCount field, it is incremented by 1 every time a user votes). When the VoteCount field value is the same as the number of users in an Event, the BufferedItem gets moved to the Item table;
	
	\item Playlist - Represents a Playlist, has a reference (URL to the playlist itself) and a type (this type describes the service that provides the playlist. Example: \textit{Spotify});
	
	\item User - Describes a User, has a Name, Username and a Phone Number;
	
	\item EventLocation - Represent the Event location. Has an address, longitude and latitude.
	
\end{itemize}


The logging database is apart of the applicational one. This is because not only might the applicational database be in a different location from the logging one but the logs can grow exponentionally, affecting storage and performance of the applicational one if the log table was in the same database.

The logging database only hosts one table on the default schema. The table is called EventLog and it represents logs. It is comprised of an EventId (type of log), LogLevel, Date, Message and StackTrace (these last two are only present in case of an exception logging):

Figure~\ref{fig:LoggingModel} illustrates the logging database model.

\begin{figure}[!ht]
	\centering
	\includegraphics[width=0.75\textwidth]{./Chapter3/Figures/Logging_Db.png}
	\caption{Data Model}
	\label{fig:LoggingModel}
\end{figure}

%\textbf{NOTE: All of these tables have a unique id that is an autoincremented/serial upon insertion integer.}

\textbf{Note:} All of these tables have a unique id that is an autoincremented/serial upon insertion integer.





%////////////////////////////////////////////////////////////////////////
%   Chapter 4
%
%////////////////////////////////////////////////////////////////////////
\chapter{Development support} 
\label{ch:Chapter4}
\vfill \minitoc \newpage

\section{Azure DevOps Services}

To ease development the group decided to use \textit{\cite{azuredevops}}. This platform provides development collaboration tools along with mechanisms that allow for a better work environment that follow DevOps practices.

Figure~\ref{fig:DevopsTasks1} illustrates DevOps Work Items (Azure Boards).

From the many tools provided by the service, the following are the ones used in this project:
\begin{itemize}
	\item Azure Boards
	\item Azure Repos
	\item Azure Pipelines
	\item Azure Artifacts
\end{itemize}


\begin{figure}[!ht]
	\centering
	\includegraphics[width=0.75\textwidth]{./Chapter4/Figures/DevopsTasks1.png}
	\caption{DevOps Boards Work Items}
	\label{fig:DevopsTasks1}
\end{figure}

\subsection{Azure Repos}
Azure Repos is a set of version control tools, allowing for multiple repositories, each one having its own version control system. For this project four repositories where created:

\begin{itemize}
	\item EventOrganizer.WebApi – Where the source code for the EventAPI and its HTTP Client is maintained.
	\item EventOrganizer.Client – Where the Xamarin client application source code is maintained.
	\item EventOrganizer.Database – Repository containing the scripts and backups for database management.
	\item EventOrganizer.Documentation – Dedicated to gather project documentation used in development.
	\item EventOrganizer.MusicApi – Contains the source code for the Music Web Api and its HTTP Client.
\end{itemize}

Figure~\ref{fig:BranchingStrategy} illustrates the feature branching strategy used.

Wanting to keep the development repositories, EventOrganizer.WebApi, EventOrganizer.MusicApi and EventOrganizer.Client organized and easy to maintain, the group decided to use the following branching strategy:

\begin{figure}[!ht]
	\centering
	\includegraphics[width=0.75\textwidth]{./Chapter4/Figures/BranchingStrategy.png}
	\caption{Feature Branching Strategy}
	\label{fig:BranchingStrategy}
\end{figure}

This branching strategy consists of creating a feature branch for each new functionality that needs to be implemented. This, combined with frequent pull requests, helped the team maintaining the source code, making sure the master branch always had a stable version and making bugs less frequent and smaller problems to resolve.

\subsection{Azure Pipelines}
Azure Pipelines is a service that can be used to build, test and deploy projects by having remote machines run a preconfigured pipeline.

For this project the group configured three pipelines:

\begin{itemize}
	\item Event Web API Pipeline – Builds the EventAPI project and deploys the Web API to Azure App Services.
	\item Generate Client Nugets – Builds and packages the EventAPI HTTP client libraries into nuget packages that contain versioning.
	\item Event Web API Pipeline – Builds the EventAPI project and deploys the Web API to Azure App Services.
	
\end{itemize}

The first two pipelines are queued for execution every time there is a new stable version, meaning every time there is a push to the master branch.

\subsection{Azure Artifacts}
Azure Artifacts gave the team the possibility of having a private nuget feed where the nuget packages generated in the Azure Pipelines are automatically pushed. This service comes with a page in Azure DevOps, where the nugets can also be viewed and managed.

\subsection{Pull Requests}

To keep code versioning efficient and organized, the decision to block direct pushes to the master branch was made.

When a feature is complete, instead of merging the feature branch to the master branch, a pull request must be open and reviewed before being completed, which results in an automatic merge to the master branch.

In addition to the mandatory review of the code, pull requests must also pass a build pipeline configured for each repository. This was done by first creating a new Azure Pipeline for each repository that builds and tests the respective solutions and secondly adding a build policy for the master branch, which specifies that the previously created pipeline should run to completion successfully.

The page for a pull request where the required policies include the previously mentioned code review, build and test run can be seen in the screen capture below.

\begin{figure}[!ht]
	\centering
	\includegraphics[width=0.75\textwidth]{./Chapter4/Figures/PullRequestExample.png}
	\caption{An example of a pull request in the development repositories}
	\label{fig:PullRequestExample}
\end{figure}

\newpage

\section{Other tools}
The group is also using Microsoft Visual Studio as it's main programming IDE as it's really flexible and it's integration with Nuget, Xamarin and UWP is seamless to the group and offers (almost) no problems and Jetbrains DataGrip as the main database IDE (although sometimes the group also uses pgAdmin for more low level operations on the PostgreSQL database).



\section{Architecture}

Figure~\ref{fig:SystemArchitecture} illustrates the system architecture.

\begin{figure}[!ht]
	\centering
	\includegraphics[width=0.75\textwidth]{./Chapter4/Figures/Architecture.png}
	\caption{System architecture.}
	\label{fig:SystemArchitecture}
\end{figure}

\subsection{Client}

The client component of this project is implemented as a mobile application, using the \textit{Android} and \textit{Windows} platform.

\subsection{Server}

The server component is implemented as a REST API, using .NET Core MVC.
\begin{itemize}	
	\item Event API - Supports the event creation and all of its operations (adding, editing and removing guests, expenses, tasks, among others);
	
	\item Music API - Supports the \textit{playlist} creation;
	
	\item Payments API (optional) - Supports the payment of the event's expenses.
	
\end{itemize}

The database operations are supported by the PostgreSQL database engine.

\subsubsection{Observations}

The motivation behind choosing this server-client architecture is the possibility of eventually creating other clients, without having to change the servers that support it.

The reason why the Payments and Media API aren't unified into the Event API is for the sake of having the benefit of the client only communicating with these APIs, yet having the possibility of comunicating with a wide range of services without having to implement them specifically.

The motivation behind the usage of SQL instead of NoSQL database types is because the entities used in this application are of relational types. Also, the atomicty and transactional behaviour that SQL databases support is of importance for consistency across event information.




%/////////////////////////////////////////////////////////////
%   Chapter 5
%   Web Client implementation
%
%/////////////////////////////////////////////////////////////

\chapter{HTR Model} 
\label{ch:Chapter5}
\vfill \minitoc \newpage

\section{Introduction}
Handwriting recognition \gls{htr} is a technology that converts handwritten text into digital format. Its purpose is to enable the efficient processing, storage, and manipulation of handwritten content through automated recognition algorithms.

\subsection{Offline HTR}
Offline handwriting recognition refers to the technology and process of converting static images or scans of handwritten text into digital text or characters. Unlike online handwriting recognition, which interprets handwriting in real time as it is being written, Offline \gls{htr} analyzes pre-existing images or documents that have been captured or scanned.

The goal of Offline \gls{htr} is to accurately recognize and convert the handwritten text in images into editable and searchable digital format. This technology finds applications in digitizing historical documents, handwritten notes, forms, and any other handwritten content that needs to be converted into machine-readable text.

The process of Offline \gls{htr} typically involves several steps:
\begin{itemize}
    \item Detection: In the detection phase, the handwritten text regions within the scanned or captured images are identified. This can be achieved through techniques such as text detection algorithms, connected component analysis, or contour-based methods. The goal is to localize and extract the regions containing the handwritten content for further processing.

    \item Image preprocessing: The scanned or captured images are enhanced, filtered, and prepared to optimize the quality of the handwriting for recognition. This may involve noise reduction, binarization, deskewing, and other techniques.
    
    \item Recognition: Machine learning algorithms, such as neural networks or Hidden Markov Models (HMM), are trained using labeled samples of handwritten characters or words. These models are then used to recognize and classify the extracted features into corresponding textual representations.

    \item Post-processing: The recognized text is further refined and processed to improve accuracy, correct errors, handle ambiguities, and align with language-specific rules and dictionaries.

\end{itemize}


\subsection{Online HTR}
Online handwriting recognition refers to the technology and process of converting handwritten input into digital text or characters in real time. It involves capturing and interpreting the movements and patterns made by a user while writing using a stylus or a digital pen on a touch-enabled device, such as a tablet or a smartphone.

Unlike offline handwriting recognition, which analyzes static images of handwritten text, online handwriting recognition takes advantage of the temporal information obtained during the writing process. This allows for real-time interpretation and immediate feedback as the user writes, making it suitable for applications where instant recognition is required, such as note-taking, electronic signature verification, form filling, and interactive whiteboards.

Online handwriting recognition systems typically use various techniques, including pattern recognition algorithms, machine learning, and neural networks, to analyze the dynamic information provided by the user's pen strokes. These algorithms analyze factors such as stroke speed, direction, pressure, and sequence to identify and interpret the handwritten characters or gestures. The recognized text can then be further processed, stored, or used in various applications and systems that require digital text input.


\section{Implementation}
In this section we will talk about the decisions made and explain the implementation of each component.
\subsection{Why Offline HTR}
In a project like this, it initially seems like a no-brainer to implement Online \gls{htr} instead of Offline \gls{htr}. However, the lack of complex yet comprehensive documentation for beginners pose a significant roadblock. Without readily available resources and clear guidance, the implementation process becomes more complex and time-consuming. This led to frustration and delays as the we struggled to navigate the complexities of Online \gls{htr} without proper guidance.

Furthermore, the absence of open-source implementations of Online \gls{htr} hinders collaboration and knowledge sharing within the community. Open-source projects often serve as valuable starting points, providing code samples, libraries, and frameworks that accelerate the development process. Without such resources, we must start from scratch or rely on proprietary solutions, limiting the ability to customize and adapt the \gls{htr} system to the specific project requirements. The potential benefits of enhanced accuracy,  real time text recognition, and dynamic input support opportunities may be overshadowed by the steep learning curve and lack of accessible resources for offline \gls{htr} implementation.

As a result, the decision to implement Offline \gls{htr} becomes less straightforward yet possible thanks to widely available open-source implementations and good guidance examples such as the one followed to implement our model \textit{\cite{HTR}}.


\subsection{Pipeline}
The \gls{htr} system follows a pipeline-based approach to process images and extract text. The pipeline consists of two main stages: text detection and text recognition.
The Figure~\ref{fig:HTRS} represents a overview of the implemented \gls{htr} system. For our needs it takes a PNG containing the postcard drawn content and returns the text as a String. 

\begin{figure}[!ht]
	\centering
	\includegraphics[trim={ 0.2cm 1cm 0cm 0cm },width=1\textwidth]{./Chapter5/Figures/HTR System}
	\caption{HTR System}
	\label{fig:HTRS}
\end{figure}

\textbf{Both components use the Tensorflow framework and are written in python.}

\subsection{Detector}
The detection is made possible by using CRAFT-Text \textit{\cite{CRAFT-Text}} detection implementation from \textit{\cite{Keras-ocr}}. The particular detector was trained for machine-generated characters based on fonts but still manages to give good results for handwritten text, big part of it's good results is the absence of visual clutter in the background as we only provide the drawn content to the model. 


The Figure~\ref{fig:Detector} illustrates how the detector works, keeping in mind that the figure is simplified for explanatory purposes and the detector detects words and not lines.

\begin{figure}[!ht]
	\centering
	\includegraphics[trim={2cm 3cm 0 1cm}, width=1.1\textwidth]{./Chapter5/Figures/Detector}
	\caption{Detection Flow}
	\label{fig:Detector}
\end{figure}

The returned value from the detector model is a list containing four points (x and y) defining a box that represents where the word is relative to the image.


\subsection{Recognizer}
The text recognition stage takes the boxes from the detector and processes them to extract the actual text content creating a temporary file for each box. It applies character segmentation, feature extraction, and sequence modeling techniques to recognize and convert the text regions into machine-readable format. A layer by layer summary based on online documentation:
\begin{itemize}
    \item Input Layer: The model takes an input image with dimensions (128, 32, 1), representing a grayscale image. It also takes input labels, which are sequences of characters.
    \item Convolutional Layers: The model starts with two convolutional blocks. Each block consists of a 2D convolutional layer followed by a max-pooling layer. The convolutional layers learn local image features and the max-pooling layers downsample the feature maps.
    \item Reshaping and Dense Layer: After the convolutional layers, the feature maps are reshaped to a new shape that is compatible with the recurrent part of the model. This reshaping operation prepares the data for input to the recurrent layers. The reshaped features pass through a fully connected dense layer with 64 units and ReLU activation.
    \item Dropout Regularization: A dropout layer is applied to reduce overfitting by randomly setting a fraction of the input units to 0 during training.
    \item Recurrent Layers: Two bidirectional LSTM layers are stacked on top of each other. Bidirectional LSTMs process the input sequence in both forward and backward directions, allowing the model to capture dependencies in both directions. The LSTM layers have dropout applied to them to prevent overfitting.
    \item Output Layer: A dense layer with a softmax activation is used as the output layer. The number of units in this layer corresponds to the vocabulary size (number of characters) plus two special tokens introduced by the CTC loss. The softmax activation produces a probability distribution over the characters.
    \item CTC Loss Layer: The output of the softmax layer is passed to the Connectionist Temporal Classification (CTC) layer. The CTC layer calculates the CTC loss, which measures the difference between the predicted sequence and the ground truth labels. It takes both the labels and the output of the softmax layer as inputs.
    \item Model Compilation: The model is compiled with the Adam optimizer. The specific learning rate and other optimizer parameters can be further customized if needed.
    \item Model Output: The output of the model is the output of the CTC layer, representing the CTC loss. 

\end{itemize}


\begin{figure}[!ht]
	\centering
	\includegraphics[trim={0cm 0cm 0 0cm}, width=1\textwidth]{./Chapter5/Figures/Recognizer}
	\caption{Recognizer Flow}
	\label{fig:Recognizer}
\end{figure}




\subsubsection{Image Preprocessing}
Image preprocessing is a critical step in offline \gls{htr} systems as it aims to enhance the quality of scanned or captured handwritten images before they are fed into the recognition model. The objective is to optimize the images for accurate recognition by applying various transformations and adjustments.

In the context of our project, the provided code snippet offers a foundational approach to image preprocessing.

Reading and Decoding Image:

The first step involves reading the image file using tf.io.read\_file and decoding it using tf.image.decode\_png.
By decoding the image as grayscale (decode\_png(image, 1)), we ensure that only the necessary information is retained.
Distortion-Free Resizing:

To achieve consistent input sizes, the distortion\_free\_resize function is employed for resizing the image to the desired dimensions (img\_size).
During resizing, the preserve\_aspect\_ratio=True parameter ensures that the original aspect ratio of the image is maintained.
This function intelligently pads the image symmetrically to ensure uniformity and prevent distortion.
Normalization:

After resizing, the image is cast to tf.float32 and normalized to a range between 0 and 1 by dividing each pixel value by 255.0.
Normalization facilitates improved convergence during training and enhances the overall performance of the recognition model.
By utilizing the preprocess\_image function, we establish a solid foundation for handling image preprocessing in our offline \gls{htr} system. This function accepts an image path as input and performs resizing, padding, and normalization operations to prepare the image for subsequent processing.


\subsubsection{Training}
The recognizer was trained using the \textit{\cite{IAM}} words dataset, which is a widely used benchmark dataset for \gls{htr} research. The IAM dataset is a comprehensive collection of handwritten English text samples contributed by different writers. It consists of 86810 training samples, 4823 validation samples and 
4823 test samples.

To get good results the recognizer was trained for roughly 60 iterations in Google's Colab Notebook servers.

\paragraph{Tweaks}
After each training iteration, the Tensorflow training process incorporates additional code through callbacks to enhance its functionality. In this particular scenario, two tweaks have been added to optimize the training process:
\begin{itemize}
    \item Early Stopping Callback:
    An early stopping callback is implemented to monitor the model's performance during training and determine if it is not improving.
    The patience parameter is set to 3, indicating that if the model does not show improvement for 3 consecutive iterations, the training process will stop early.
    By setting restore\_best\_weights to True, the callback ensures that the best weights achieved during training are restored before stopping, allowing for optimal model performance.
    The verbose parameter is set to 1, enabling the callback to display informative messages about its operations. 
    \item Checkpoint Callback:
    A checkpoint callback is added to periodically save the model's weights during training.
    The filepath parameter specifies the path where the weights will be saved.
    By setting save\_weights\_only to True, only the weights of the model will be saved, reducing storage requirements.
    The verbose parameter is set to 1, enabling the callback to display informative messages about the saving process.
\end{itemize}

These tweaks improve the training process by introducing early stopping criteria based on the model's performance and by providing checkpoints to save the model's weights at different stages.


\subsubsection{Postprocessing}
To be able to decode the output produced by the Output Layer (dense 2) a function called  \textit{decode\_batch\_predictions} is implemented in the guide \textit{\cite{HTR}}.

The function takes the model predictions pred as input. These predictions are usually in the form of probability distributions over the characters in the vocabulary for each time step.

The variable input\_len is created to specify the length of the input sequences for each prediction in the batch. It is set to be the same for all predictions and is equal to the number of time steps in the predictions.

The function utilizes the CTC (Connectionist Temporal Classification) decoding method to convert the predictions into sequences of characters. It applies the ctc\_decode function from Keras backend, passing the predictions, input length, and using greedy search (other methods like beam search can be used for more complex tasks). The ctc\_decode function returns the decoded sequences.

The results variable stores the decoded sequences. It selects the first element [0][0] from the ctc\_decode output, which represents the decoded labels for the batch. It also truncates the sequences to a maximum length max\_len if necessary.

The function iterates over each decoded sequence in results. For each sequence, it applies several operations to convert the numerical labels to actual text.

First, it uses tf.where to find the positions where the labels are not equal to -1 (a special token often used in CTC decoding to represent blank or no label).

It then uses tf.gather to gather the non-equal elements from the labels.

The gathered labels are passed through num\_to\_char function, which maps the numerical labels to their corresponding characters.

Next, tf.strings.reduce\_join is applied to concatenate the characters into a single string representation.

Finally, numpy().decode("utf-8") is used to convert the string from a TensorFlow tensor to a regular Python string, and the resulting string is appended to the output\_text list.

After iterating over all the sequences in results, the function returns the output\_text list, which contains the decoded texts for each prediction in the batch.

In summary, the decode\_batch\_predictions function takes the model predictions, performs CTC decoding to convert the predictions into sequences of characters, and applies additional post-processing steps to obtain the final text representations for the predictions.


\subsection{Natural Text Ordering}
One of the challenges in \gls{htr} is maintaining the natural ordering of text when dealing with multi-line or multi-column documents. To address this challenge, we came up with a algorithm that uses the average box size to calculate a margin of error for each point in a line.
In our implementation there is always two fixed boxes Figure~\ref{fig:PFBoxes}

\begin{figure}[!ht]
	\centering
	\includegraphics[trim={0cm 0cm 0 0cm}, width=1\textwidth]{./Chapter5/Figures/Main Sections Postcard}
	\caption{Postcard Main Sections}
	\label{fig:PFBoxes}
\end{figure}

\newpage

The ordering algorithm works like this:

We start by calculating the height of every box, using vector calculation. We do this for every box and divide by the number of boxes obtaining the average box height.

\begin{figure}[!ht]
	\centering
	\includegraphics[trim={0cm -0.5cm 0 -0.5cm}, width=0.7\textwidth]{./Chapter5/Figures/Height Box}
	\caption{Average height calculation for left section}
	\label{fig:AVGH}
\end{figure}


Now all we have to do is pick the same located point in every box (top left or right left, etc...) and test the y value +- the calculated average box/2. If the y value is inside the range y-average height/2 to y+average height/2 then compare the x values else compare the y values. A lower x value means it comes earlier in the natural ordering. A higher y value than the current one means it comes later in the natural ordering. 

\section{Limitations}
IAM dataset vocabulary is primarily focused on the English language. It includes a wide range of alphanumeric characters, including uppercase and lowercase letters (A-Z, a-z), digits (0-9), and common punctuation marks. However, it does not cover the entire spectrum of possible characters that can exist in different languages or writing systems.

This vocabulary limitation means that the IAM dataset may not be suitable for recognizing text in languages other than English or for dealing with specialized symbols or characters that are outside the dataset's predefined set. For example, if the dataset does not include characters specific to a particular language or domain, the \gls{htr} model trained on IAM may struggle to accurately recognize and transcribe such characters.

While having made all the possible optimizations for using the detector, if the user draws text right next to each other there's a good change it will detect it all as a whole word.


%/////////////////////////////////////////////////////////////
%   Chapter 6
%   Results
%   
%
%/////////////////////////////////////////////////////////////
\chapter{Client} 
\label{ch:Chapter6}
\vfill \minitoc \newpage

The client was implemented in Android and uses Android's Jetpack Compose UI toolkit.

\begin{figure}[!ht]
	\centering
	\includegraphics[trim={0cm 0cm 0 0cm}, width=1\textwidth]{./Chapter6/Figures/Android NavGraph}
	\caption{Android Activity Navigation Graphic}
	\label{fig:NavgGraph}
\end{figure}


\section{Android System and Compose Framework Overview}
Before showing the client implementation its best to give some context and basic knowledge about the android system and the Compose framework.

\subsection{Android Manifest}
The AndroidManifest.xml file is an essential configuration file in Android development that provides essential information about the Android application to the Android system. It is located in the root directory of the Android project and is required for every Android application.

The Android manifest file contains important metadata about the app, including its package name, version number, permissions, activities, services, broadcast receivers, and more. It serves as a blueprint for the Android system to understand the structure and behavior of the application.

\subsection{Android Activity}
In the context of Android app development, an Activity is a fundamental component of an Android application that represents a single screen with a user interface. It is a crucial part of the overall app architecture and is responsible for handling user interactions and presenting visual elements to the user.

An Activity acts as a container for the user interface and provides a window in which the app's UI elements, such as buttons, text fields, images, and other widgets, are displayed. It manages the lifecycle of these UI components and handles user input events, such as button clicks or touch gestures.

Each Activity has a corresponding Java or Kotlin class that extends the Activity base class or its subclasses provided by the Android framework. This class contains methods that define the behavior of the Activity during different stages of its lifecycle, such as creation, starting, pausing, resuming, stopping, and destruction.

When an app is launched, typically, the main Activity is created and displayed to the user. The Activity is responsible for setting up the initial UI layout, interacting with data sources (e.g., retrieving data from a database or an API), and responding to user actions. It can also communicate with other Activities, such as starting a new Activity for a different screen or receiving results from a previously started Activity.

\subsection{Data Storing}
Android uses a file system that's similar to disk-based file systems on other platforms. The system provides several options for you to save your app data:
\begin{itemize}
    \item App-specific storage: Store files that are meant for your app's use only, either in dedicated directories within an internal storage volume or different dedicated directories within external storage. Use the directories within internal storage to save sensitive information that other apps shouldn't access;
    \item Shared storage: Store files that your app intends to share with other apps, including media, documents, and other files;
    \item Preferences: Store private, primitive data in key-value pairs;
    \item Databases: Store structured data in a private database using the Room persistence library.    
\end{itemize}


\subsection{ViewModel}
The ViewModel class is a business logic or screen level state holder. It exposes state to the UI and encapsulates related business logic. Its principal advantage is that it caches state and persists it through configuration changes. This means that your UI doesn’t have to fetch data again when navigating between activities, or following configuration changes, such as when rotating the screen.

\subsection{Canvas}
The Android Canvas is a fundamental graphics component provided by the Android framework. It serves as a drawing surface onto which we can render custom graphics, shapes, images, and text. The Canvas provides a set of drawing methods that allow us to create and manipulate visual elements within an Android application.

When working with the Canvas, we can perform various operations such as drawing lines, rectangles, circles, arcs, and paths. We can also apply transformations like translation, rotation, scaling, and skewing to manipulate the position and orientation of the drawn elements. Additionally, the Canvas supports the rendering of text, allowing us to display custom text with different fonts, sizes, colors, and styles.

\subsection{Making HTTP Request}
When developing Android applications, it is common to interact with web services and APIs to retrieve data or send data to a server. One popular library for making HTTP requests in Android is OkHttp.

\subsubsection{OkHttp}
OkHttp is an open-source HTTP client library for Java and Android applications. It is developed by the same team behind the widely-used Retrofit library and offers a simple and efficient way to make HTTP requests and handle responses. OkHttp is built on top of the Java standard library's HttpURLConnection, providing a more convenient and powerful API.

\newpage


\section{Activities}
In this section we will demonstrate all activities implemented.
\subsection{Permissions}
The Permissions activity handles all requests to permissions needed for the application to work.
The Application needs permission to read contacts.

Figures \ref{fig:PA} and \ref{fig:PA1} illustrate the implemented activity. 

\begin{figure}[!ht]
	\centering
	\includegraphics[trim={0cm 0cm 0 0cm}, width=0.4\textwidth]{./Chapter6/Figures/Permission}
	\caption{Permission Activity}
	\label{fig:PA}
\end{figure}


\begin{figure}[!ht]
	\centering
	\includegraphics[trim={0cm 0cm 0 0cm}, width=0.4\textwidth]{./Chapter6/Figures/Permission access}
	\caption{Prompt Permission Activity}
	\label{fig:PA1}
\end{figure}

\newpage

\subsection{Sign In}

The Sign-in Activity serves as the component for managing the user's authentication process, containing both logging in and registering in the service. Additionally, it ensures the secure storage of the user's token by leveraging the EncryptedSharedPreferences.

Upon launching the Sign-in Activity, users are presented with a user-friendly interface where they can enter their credentials or choose to register as a new user. The activity handles the input validation and securely communicates with the server-side authentication API.

Once the user's credentials are verified, the Sign-in Activity retrieves the authentication token from the server's response. To ensure the token's confidentiality, it is stored using the EncryptedSharedPreferences. This specialized SharedPreferences implementation employs encryption algorithms to protect sensitive data from unauthorized access.

By utilizing the EncryptedSharedPreferences, the Sign-in Activity safeguards the user's authentication token, preventing it from being tampered with or exposed. This secure storage mechanism provides an additional layer of protection for user data, mitigating the risks associated with unauthorized token access.


In addition, the Sign-in Activity incorporates an automatic phone number region retrieval feature by leveraging the Carrier information.

Figures \ref{fig:SA1}, \ref{fig:SA2} and \ref{fig:SA3} illustrate the implemented activity.


\begin{figure}[!ht]
	\centering
	\includegraphics[trim={0cm -3cm 0 -3cm}, width=0.4\textwidth]{./Chapter6/Figures/Login}
	\caption{Signin Activity}
	\label{fig:SA1}
\end{figure}

\newpage

It also does local verification's to user's input. Password and Phone Number validations are done.

\begin{figure}[!ht]
	\centering
	\includegraphics[trim={0cm -3cm 0 -3cm}, width=0.4\textwidth]{./Chapter6/Figures/Login Invalid Password 1}
	\caption{Signin Activity invalid password size}
	\label{fig:SA2}
\end{figure}

\newpage


\begin{figure}[!ht]
	\centering
	\includegraphics[trim={0cm -3cm 0 -3cm}, width=0.4\textwidth]{./Chapter6/Figures/Login Invalid Password 2}
	\caption{Signin Activity missing invalid password}
	\label{fig:SA3}
\end{figure}

\newpage

\subsection{Home}
The Home Activity serves as a central hub for connecting to the web API and retrieving essential information related to registered users, messages, and chats. Its primary purpose is to display all chats in a user-friendly manner, with the chats ordered based on the most recent message received.

By establishing a connection with the web API, the Home Activity can fetch the necessary data to populate the chat interface. It retrieves information about registered users, ensuring that the appropriate user profiles are displayed within the chat list. Additionally, it retrieves messages associated with each chat, allowing users to view their conversation history.

The Home Activity organizes the chats in a manner that prioritizes the most recent interactions. By ordering the chats based on the last message received, users can quickly identify and access their most recent conversations.

Moreover, the Home Activity provides intuitive controls and a simple interface, users can create new chat groups, within a button.

Figure~\ref{fig:HA1} illustrate the implemented activity.

\newpage

\begin{figure}[!ht]
	\centering
	\includegraphics[trim={0cm -3cm 0 -3cm}, width=0.4\textwidth]{./Chapter6/Figures/Home}
	\caption{Home Activity}
	\label{fig:HA1}
\end{figure}

\newpage

\newpage

\subsection{Create Chat}
The CreateChat activity searches for the users stored in the local database and lets you pick the phone numbers you want to add to the chat. Every chat needs a name so a popup dialog input message shows when clicking the check button.

Figures \ref{fig:CCA1} and \ref{fig:CCA2} illustrate the implemented activity.


\begin{figure}[!ht]
	\centering
	\includegraphics[trim={0cm -3cm 0 -3cm}, width=0.4\textwidth]{./Chapter6/Figures/Picked Users}
	\caption{CreateChat Activity}
	\label{fig:CCA1}
\end{figure}

\newpage

\begin{figure}[!ht]
	\centering
	\includegraphics[trim={0cm -3cm 0 -3cm}, width=0.4\textwidth]{./Chapter6/Figures/Create Chat Name}
	\caption{CreateChat Activity name prompt}
	\label{fig:CCA2}
\end{figure}

\subsection{Chat View}
The Chat activity obtains the information about the current chat messages. 
It displays the postcards in order by the timestamp and above the same it shows the number from the person that sent the message. In the future this will be changed to query users in the local database and get their name.  

Figures \ref{fig:CVA1}, \ref{fig:CVA2} , \ref{fig:CVA3} and \ref{fig:CVA4} illustrate the implemented activity.

\begin{figure}[!ht]
	\centering
	\includegraphics[trim={0cm -3cm 0 -3cm}, width=0.4\textwidth]{./Chapter6/Figures/Chat}
	\caption{Chat Activity}
	\label{fig:CVA1}
\end{figure}

\newpage

\begin{figure}[!ht]
	\centering
	\includegraphics[trim={0cm -3cm 0 -3cm}, width=0.4\textwidth]{./Chapter6/Figures/Chat Bottom Sheet}
	\caption{Chat Activity bottom templates list}
	\label{fig:CVA2}
\end{figure}


\newpage

\begin{figure}[!ht]
	\centering
	\includegraphics[trim={0cm -3cm 0 -3cm}, width=0.4\textwidth]{./Chapter6/Figures/Chat Pick Template}
	\caption{Chat Activity pick template}
	\label{fig:CVA3}
\end{figure}

On clicking the edit button the user is sent to the Draw Activity.

\newpage

\begin{figure}[!ht]
	\centering
	\includegraphics[trim={0cm -3cm 0 -3cm}, width=0.4\textwidth]{./Chapter6/Figures/Sent Postcard}
	\caption{CreateChat sent postcard}
	\label{fig:CVA4}
\end{figure}

\bigskip
\bigskip
\bigskip
\bigskip
\bigskip
\bigskip


This is a beta version some bugs still need to be fixed.


\newpage


\subsection{Draw Postcard}
The Draw activity plays a crucial role in allowing users to edit and personalize a postcard using the Canvas. It involves implementing complex code to enable features such as drawing, zooming, and saving the postcard.

The challenge in implementing drawing and zoom features arise from the absence of a built-in two-finger zoom and one-finger touch draw function in the compose toolkit. To overcome this limitation, an in-depth analysis of the compose toolkit's inner code was conducted. By examining the underlying mechanisms of the compose toolkit, the necessary functionality for drawing and zooming was achieved.

To facilitate the saving of the canvas, a list is used to store the properties of each path. Each path property consists of the actual path data and the stroke style applied to it. When the canvas needs to be saved, the application iterates through each path in the list and generates an SVG (Scalable Vector Graphics) file.

During the SVG file creation process, the application adds the necessary path movements and stroke styles to accurately represent each drawn path. This ensures that the saved SVG file faithfully represents the visual appearance of the canvas.

Figures \ref{fig:DA1} and \ref{fig:DA2} illustrate the implemented activity.



\begin{figure}[!ht]
	\centering
	\includegraphics[trim={0cm -3cm 0 -3cm}, width=0.4\textwidth]{./Chapter6/Figures/Draw Postcard}
	\caption{Draw Activity}
	\label{fig:DA1}
\end{figure}

\newpage

\begin{figure}[!ht]
	\centering
	\includegraphics[trim={0cm -3cm 0 -3cm}, width=0.4\textwidth]{./Chapter6/Figures/Postcard Portrait}
	\caption{Draw Activity draw}
	\label{fig:DA2}
\end{figure}

\newpage


\subsection{View Postcard}
The View Postcard Activity provides users with a dedicated interface to view postcards and offers the ability to save them with a simple button press. This activity focuses on delivering a seamless and immersive experience for users to enjoy the postcard content.

Figure~\ref{fig:VA} illustrate the implemented activity.

\begin{figure}[!ht]
	\centering
	\includegraphics[trim={0cm -3cm 0 -3cm}, width=0.4\textwidth]{./Chapter6/Figures/Postcard Save}
	\caption{Postcard View}
	\label{fig:VA}
\end{figure}




%/////////////////////////////////////////////////////////////
%   Chapter 8
%
%   Conclusions and Future Work
%
%/////////////////////////////////////////////////////////////
\chapter{Conclusions and Future Work} 
\label{ch:Chapter8}
\vfill \minitoc \newpage

\section{About Xamarin}

While Xamarin is a great toolkit to develop cross platform applications that reuse a lot of the code and view definitions (especially Xamarin Forms) between devices the group realized that there is some problems with the implementation of the technology and the documentation available for it. 
\newline
For example, upon using WebView there is a bug on the Navigation mechanism of the page that sometimes occurs in an unhandled null reference exception while trying to "Pop" to an earlier calling Content Page, this has been experienced by many users on the UWP and Android implementations of this type of Page.
\newline
Also the group had some troubles trying to test the authentication mechanism on the client application because as far as the group's research went there is no way to avoid having a valid certificate when handling OAuth2 requests on Xamarin (not even local trusted certificates). So the solution was deploying the app on Azure since this hosting provides a valid certificate.


\section{About .NET Core}

.NET Core is a great free and open-source framework that allows cross-platform software development. The documentation is solid and the group felt at home coding the API using this technology coming from a .NET Framework coding background.

\section{About PostgreSQL}
PostgreSQL is a great SQL database engine that is really similar in usage to Microsoft's SQL Server but it is completely free and open source and it has a huge community supporting it. However the group had to host their own PostgreSQL server on a personal computer until it found out that \cite{elephantsql} provides a free service of a PostgreSQL instance with 20Mb of space (which is enough for developments on this project).


\section{Future Developments}

After making the current API and Client Application more stable the group needs to get a better, more solid hosting mechanism (as of now it is a mix between Azure and the groups personal computers). The next development so the project is complete is the voting system and the Music API(the platform to unify music services to played at events). Development is currently approximately 70\% done.














% ///////////////////////////////////////////////////////
% Configuracao do header para apendices e referencias
% ///////////////////////////////////////////////////////

% Formato da pagina esquerda (par): <Pagina><Capitulo nr Nome>
\fancyhead[LE]{\small\thepage\hspace{3em}\nouppercase{\small\leftmark}}
\fancyhead[RE,LO]{}
% Formato da pagina direita (impar): <Numero da Seccao> <Nome da seccao> <Pagina>
\fancyhead[RO]{\nouppercase{\small\rightmark}\hspace{3em}\small\thepage}


%% The Appendices part is started with the command \appendix;
%% appendix sections are then done as normal sections
%% \appendix

%% \section{}
%% \label{}

%\appendix
%
%\renewcommand\thesection{\Alph{section}}

\renewcommand\chaptername{Appendix}

% Use:
%\appendix
%
% Or use:
\begin{appendices}

%\adjustmtc

%/////////////////////////////////////////////////////////////
%   Appendix A
% 
%   
%
%/////////////////////////////////////////////////////////////
\chapter{Other Definitions}
\label{app:AppendixA}
%\vfill \minitoc \newpage



\section{Technologies}
\begin{itemize}
	\item Spring Boot and MVC - Open Source Framework for Web Applications;
	\item JVM - Java Virtual Machine;
    \item Tensorflow - TensorFlow is a framework to create machine learning models for desktop, mobile, web, and cloud.
	\item Libphonenumber - Library to validate phone number format;
	\item Jetpack Compose - Android’s recommended toolkit for building native UI;
	\item Kotlin - Programming Language that extends Java base language;
\end{itemize}
	


%/////////////////////////////////////////////////////////////
%   Appendix B
% 
%   
%
%/////////////////////////////////////////////////////////////
%\chapter{Running times}
\label{app:AppendixB}
%\vfill \minitoc \newpage




%/////////////////////////////////////////////////////////////
%   Appendix C
% 
%   
%
%/////////////////////////////////////////////////////////////
%\chapter{Developed Software}
\label{app:AppendixC}
%\vfill \minitoc \newpage




\end{appendices}


% -------------------------------------------------------------
%   Bibliography
% -------------------------------------------------------------
%% Numbered
%\bibliographystyle{elsarticle-num} % Elsevier template (See Elsevier sample file)

%% APA style
\bibliographystyle{model5-names}

\bibliography{References/References}


\end{document}


